\documentclass{article}
\usepackage[utf8]{inputenc}
\usepackage{kotex}
\usepackage{listings}
\usepackage{color}

\definecolor{dkgreen}{rgb}{0,0.6,0}
\definecolor{gray}{rgb}{0.5,0.5,0.5}
\definecolor{mauve}{rgb}{0.58,0,0.82}

\lstset{frame=tb,
  language=Java,
  aboveskip=3mm,
  belowskip=3mm,
  showstringspaces=false,
  columns=flexible,
  basicstyle={\small\ttfamily},
  numbers=none,
  numberstyle=\tiny\color{gray},
  keywordstyle=\color{blue},
  commentstyle=\color{dkgreen},
  stringstyle=\color{mauve},
  breaklines=true,
  breakatwhitespace=true,
  tabsize=3
}
\lstset{language=java}

\title{자바프로그래밍및실습 과제 2}
\author{214823 박종현}
\date{April 2022}

\begin{document}

\maketitle

\pagebreak

\section{실습 과제}
\subsection{실습 과제 1}
\begin{lstlisting}
import java.util.Scanner;

public class Main {
  public static void main(String args[]) {
    Scanner sc = new Scanner(System.in);
    System.out.print("정수를 입력하시오 >> ");
    int n = sc.nextInt(); // 입력
    for (int i = n; i > 0; i--) { // n회 반복
      System.out.println("*".repeat(i)); // repeat 메서드는 인자만큼 문자열 객체를 반복함
      // for (int j = 0; j < i; j++) System.out.print("*");
      // System.out.println();
    }
  }
}
\end{lstlisting}
\subsection{실습 과제 2}
\begin{lstlisting}
import java.util.Scanner;

public class Main {
  public static void main(String args[]) {
    Scanner sc = new Scanner(System.in);
    System.out.print("정수를 입력하시오 >> ");
    int n = sc.nextInt();
    // n만큼 반복
    for (int i = 0; i < n; i++) {
      System.out.println("*".repeat(2*i+1)); // 별이 2*i+1회 찍혀야함 (i = 반복 횟수, 0부터 시작)
      // for (int j = 0; j < 2*i+1; j++) System.out.print("*");
      // System.out.println();
    }
  }
}
\end{lstlisting}
\subsection{실습 과제 3}
\begin{lstlisting}
import java.util.Scanner;

public class Main {
  public static void main(String args[]) {
    Scanner sc = new Scanner(System.in);
    System.out.print("소문자 알파벳 하나를 입력하시오 >> ");
    // charAt 메서드는 인자로 받아오는 인덱스 번호를 통해 문자열을 인덱싱할 수 있도록 함
    // (= String 형 값을 char 값으로 캐스팅)
    char gets = sc.next().charAt(0);
    // 'a'는 97임, gets에서 시작하여 'a'까지 반복
    for (int i = gets; i >= 97; i--) { 
      // 각 열마다 알파벳 출력
      for (int j = 97; j <= i; j++) {
        System.out.print((char)j);
      }
      System.out.println(); // 개행
    }
  }
}
\end{lstlisting}
\subsection{실습 과제 4}
\begin{lstlisting}
import java.util.Scanner;
import java.lang.Math;

public class Main {
  public static void main(String args[]) {
    Scanner sc = new Scanner(System.in);
    // 문제 제시 값
    int n = (int)(Math.random()*50.0);
    int tries = 0; // 시도 횟수 기록
    while (true) { // 무한 루프
      tries++;
      System.out.print("숫자를 추측하여 보세요: ");
      int gets = sc.nextInt(); // 입력
      if (gets < n) System.out.println("UP");
      else if (gets > n) System.out.println("DOWN");
      else {
        // 정답일 경우 시도 횟수를 출력하고 반복문 종료
        System.out.println(String.format("정답입니다. 시도횟수 = %d", tries));
        break;
      }
    }
  }
}
\end{lstlisting}
\section{과제}
\subsection{과제 1}
\begin{lstlisting}
import java.util.Scanner;

public class Main {
  public static void main(String[] args) {
    int gets = -1; // do-while문을 사용하지 않는 대신 음수로 초기화
    Scanner sc = new Scanner(System.in);
    System.out.print("양의 정수를 입력하시오: ");
    while (gets <= 0) { // 양의 정수가 입력될 때까지 반복
      gets = sc.nextInt();
      if (gets > 0) break; // 입력값이 양수이면 아래 라인을 수행하지 않고 반복 종료
      System.out.print("양의 정수가 아닙니다. 다시 입력하세요: ");
    }
    System.out.println(); // 개행
    // 출력 내용을 덜 복잡하게 확인할 수 있도록 문자열 포매팅 수행
    System.out.println(String.format("%d의 약수는 다음과 같습니다.", gets));

    // 가장 러프한 형태의 약수 알고리즘 구현
    // 시간복잡도: O(n)
    for (int i = 1; i <= gets; i++) {
      if (gets % i == 0) System.out.print(String.format("%d ", i));
    }

    // 터미널 환경에서 프로세스를 실행하면
    // 종료 시 개행되어야 다음 터미널 명령을 깔끔하게 입력할 수 있음.
    System.out.println();
  }
}
\end{lstlisting}
\subsection{과제 2}
\begin{lstlisting}
import java.util.Scanner;

public class Main {
  public static boolean primes[];
  public static void main(String[] args) {
    int gets = -1; // do-while문을 사용하지 않는 대신 음수로 초기화
    Scanner sc = new Scanner(System.in);
    System.out.print("양의 정수를 입력하시오: ");
    while (gets <= 0) { // 양의 정수가 입력될 때까지 반복
      gets = sc.nextInt();
      if (gets > 0) break; // 입력값이 양수이면 아래 라인을 수행하지 않고 반복 종료
      System.out.print("양의 정수가 아닙니다. 다시 입력하세요: ");
    }
    System.out.println(); // 개행

    // 에라토스테네스의 체
    // 시간복잡도: O(n log log n)
    primes = new boolean[gets+1]; // gets까지 인덱싱할 수 있도록 gets+1 크기의 부울 배열 생성
    for (int i = 0; i < gets+1; i++) primes[i] = true; // 부울 배열 기본 초기값은 false이므로 true로 초기화
    primes[0] = false; primes[1] = false; // 0과 1은 소수가 아님
    System.out.print("결과: ");
    for (int i = 2; i < gets+1; i++) {
      if (primes[i]) { // 만약 primes[i]가 true라면 i가 소수임을 의미함
        System.out.print(String.format("%d ", i));
        for (int j = i; j < gets+1; j+=i) primes[j] = false; // i의 배수를 전부 false(합성수)로 처리
      }
    }

    // 터미널 환경에서 프로세스를 실행하면
    // 종료 시 개행되어야 다음 터미널 명령을 깔끔하게 입력할 수 있음.
    System.out.println();
  }
}
\end{lstlisting}
\subsection{과제 3}
\begin{lstlisting}
import java.util.Scanner;

public class Main {
  public static void main(String[] args) {
    Scanner sc = new Scanner(System.in);
    int gets = -(int)2e9, max = gets; // gets를 입력, max를 최댓값으로 지정, -INF로 초기화
    String[] arr = {"첫", "두", "세"}; // 문장 일부분을 배열로 초기화
    for (String var: arr) { // foreach 구문 사용, arr.length 만큼 반복
      System.out.print(var+"번째 정수를 입력하세요: "); // 문장 병합
      gets = sc.nextInt(); // 입력
      max = max > gets ? max : gets; // max값 갱신
    }
    System.out.println();
    System.out.println(String.format("Max값은 %d입니다.", max));
  }
}
\end{lstlisting}
\begin{lstlisting}
import java.util.Scanner;

public class Main {
  public static void main(String[] args) {
    Scanner sc = new Scanner(System.in);
    int gets = (int)2e9, min = gets; // gets를 입력, max를 최댓값으로 지정, INF로 초기화
    String[] arr = {"첫", "두", "세"}; // 문장 일부분을 배열로 초기화
    for (String var: arr) { // foreach 구문 사용, arr.length 만큼 반복
      System.out.print(var+"번째 정수를 입력하세요: "); // 문장 병합
      gets = sc.nextInt(); // 입력
      min = min < gets ? min : gets; // max값 갱신
    }
    System.out.println();
    System.out.println(String.format("Min값은 %d입니다.", min));
  }
}
\end{lstlisting}
\subsection{과제 4}
\begin{lstlisting}
public class Main {
  public static void main(String[] args) {
    // 문제 제시 값
    int n[][] = {{1},{1,2,3},{1},{1,2,3,4},{1,2}};

    // 배열을 행렬로 생각했을 때
    // 첫번째 for문으로 열 조회
    for (int i = 0; i < n.length; i++) {
      // 두번째 for문으로 행 조회
      for (int j = 0; j < n[i].length; j++) {
        // (i, j)번 요소 조회
        System.out.print(String.format("%d ", n[i][j]));
      }
      // 하나의 열을 모두 조회하면 개행
      System.out.println();
    }
  }
}
\end{lstlisting}
\subsection{과제 5}
\begin{lstlisting}
import java.util.Scanner;

public class Main {
  public static void main (String[] args) {
    Scanner sc = new Scanner(System.in);
    
    // 문제 제시 값 
    int[] unit = {50000, 10000, 1000, 500, 100, 50, 10};

    // 잔액 변수 생성
    int remains = -1;
    System.out.print("금액을 입력하시오 >> ");
    remains = sc.nextInt();
    for (int u: unit) { // foreach문 사용, u는 unit의 요소(단위)
      int n = remains / u; // 잔액을 단위로 나눈 몫 구하기
      if (n == 0) continue; // 몫이 0이면 다음 단위 처리
      System.out.println(String.format("%d원: %d개", u, n));
      remains -= u*n;
    }
  }
}
\end{lstlisting}
\subsection{과제 6}
\begin{lstlisting}
import java.lang.Math;

public class Main {
  public static void main(String[] args) {
    // 배열 초기화
    int[][] arr = new int[4][4];
    // 이중 포문으로 배열의 각 값 접근
    for (int i = 0; i < 4; i++) {
      for (int j = 0; j < 4; j++) {
        arr[i][j] = (int)(Math.random()*10+1); // 문제 제시 값
      }
    }
    for (int i = 0; i < 4; i++) {
      for (int j = 0; j < 4; j++) {
        // 행 출력
        System.out.print(String.format("%d ", arr[i][j]));
      }
      // 하나의 열을 모두 출력하면 개행
      System.out.println();
    }
  }
}
\end{lstlisting}
\subsection{과제 7}
\begin{lstlisting}
import java.util.Scanner;

public class Main {
  public static void main(String[] args) {
    Scanner sc = new Scanner(System.in);
    // 값 초기화
    int gets = -1, arr[];
    System.out.print("정수 몇 개? >> ");
    gets = sc.nextInt();
    // 입력값만큼의 크기를 가진 배열 생성
    arr = new int[gets];
    
    // 초기화
    for (int i = 0; i < gets; i++) arr[i] = (int)(Math.random()*100+1);
    
    // 각 열에서 몇개 값을 출력했는지 기록함
    int lnCounts = 0;
    for (int i = 0; i < gets; i++) {
      System.out.print(String.format("%d ", arr[i]));
      lnCounts++;
      // 만약 깂을 10번 출력했다면 개행
      if (lnCounts % 10 == 0) System.out.println();
    }

    // 터미널 환경에서 프로세스를 실행하면
    // 종료 시 개행되어야 다음 터미널 명령을 깔끔하게 입력할 수 있음.
    System.out.println();
  }
}
\end{lstlisting}

\end{document}
