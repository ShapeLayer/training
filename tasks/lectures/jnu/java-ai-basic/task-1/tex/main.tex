\documentclass{article}
\usepackage[utf8]{inputenc}
\usepackage{kotex}
\usepackage{listings}
\usepackage{color}
\definecolor{dkgreen}{rgb}{0,0.6,0}
\definecolor{gray}{rgb}{0.5,0.5,0.5}
\definecolor{mauve}{rgb}{0.58,0,0.82}

\lstset{frame=tb,
  language=Java,
  aboveskip=3mm,
  belowskip=3mm,
  showstringspaces=false,
  columns=flexible,
  basicstyle={\small\ttfamily},
  numbers=none,
  numberstyle=\tiny\color{gray},
  keywordstyle=\color{blue},
  commentstyle=\color{dkgreen},
  stringstyle=\color{mauve},
  breaklines=true,
  breakatwhitespace=true,
  tabsize=3
}

\title{JAVA 프로그래밍및실습 #1 과제 #1}
\author{박종현}
\date{April 2022}

\begin{document}

\maketitle

\section{자바 실습 환경 설치 및 Hello2030 출력}

\begin{lstlisting}
\lstset{language=Java}
public class Hello2030 {
	public static void main(String[] args) {
		int n = 2030;
		System.out.println("헬로"+n);
	}
}
\end{lstlisting}

\section{예제 2.x 소스코드 각각 작성}
\subsection{}
\begin{lstlisting}
\lstset{language=Java}
public class Prob21 {

	public static int sum(int n, int m) {
		return n + m;
	}
	
	public static void main(String[] args) {
		int i = 20;
		int s;
		char a;
		
		s = sum(i, 10);
		a = '?';
		System.out.println(a);
		System.out.println("Hello");
		System.out.println(s);
	}

}
\end{lstlisting}
\subsection{}
\begin{lstlisting}
\lstset{language=Java}
public class Prob22 {
	public static void main(String[] args) {
		final double PI = 3.14;
		double radius = 10.0;
		double circleArea = radius * radius * PI;
		System.out.println("원의 면적 = " + circleArea);
	} 
}
\end{lstlisting}
\subsection{}
\begin{lstlisting}
\lstset{language=Java}
public class Prob2_3 {
  public static void main(String[] args) {
    byte b = 127;
    int i = 100;
    System.out.println(b+i);
    System.out.println(10/4);
    System.out.println(10.0/4);
    System.out.println((char)0x12340041);
    System.out.println((byte)(b+i));
    System.out.println((int)2.9 + 1.8);
    System.out.println((int)(2.9+ 1.8));
    System.out.println((int)2.9 + (int)1.8);
  }
}
\end{lstlisting}
\subsection{}
\begin{lstlisting}
\lstset{language=Java}
import java.util.Scanner;
public class Prob2_4 {
  public static void main(String args[]) {
    System.out.println("이름, 도시, 나이, 체중, 독신 여부를 빈칸으로 분리하여 입력하세요.");
    Scanner scanner = new Scanner(System.in);

    String name = scanner.next();
    System.out.print("이름은 " + name + ", ");
    
    String city = scanner.next();
    System.out.print("도시는 " + city + ", ");

    int age = scanner.nextInt();
    System.out.print("나이는 " + age + "살, ");

    double weight = scanner.nextDouble();
    System.out.print("체중은 " + weight + "kg, ");

    boolean isSingle = scanner.nextBoolean();
    System.out.print("독신 여부는 " + isSingle + "입니다.");
    scanner.close();
  }
}
\end{lstlisting}
\subsection{}
\begin{lstlisting}
\lstset{language=Java}
import java.util.Scanner;

public class Main {
  public static void main(String args[]) {
    Scanner sc = new Scanner(System.in);
    
    System.out.print("정수를 입력하세요: ");
    int time = sc.nextInt();
    int second = time % 60;
    int minute = (time / 60) % 60;
    int hour = (time / 60) / 60;

    System.out.print(time + "초는 ");
    System.out.print(hour + "시간, ");
    System.out.print(minute + "분, ");
    System.out.println(second + "초입니다.");
    sc.close();
  }
}
\end{lstlisting}
\subsection{}
\begin{lstlisting}
\lstset{}
public class Main {
  public static void main(String[] args) {
    int a = 3, b = 3, c = 3;
    a += 3;
    b *= 3;
    c %= 2;
    System.out.println("a=" + a + ", b=" + b + ", c=" + c);

    int d = 3;
    a = d++;
    System.out.println("a=" + a + ", d=" + d);
    a = ++d;
    System.out.println("a=" + a + ", d=" + d);
    a = d--;
    System.out.println("a=" + a + ", d=" + d);
    a = --d;
    System.out.println("a=" + a + ", d=" + d);
  }
}
\end{lstlisting}
\subsection{}
\begin{lstlisting}
\lstset{language=Java}
public class Main {
  public static void main(String[] args) {
    System.out.println('a' > 'b');
    System.out.println(3 >= 2);
    System.out.println(-1 < 0);
    System.out.println(3.45 <= 2);
    System.out.println(3 == 2);
    System.out.println(3 != 2);
    System.out.println(!(3 != 2));

    System.out.println((3 > 2) && (3 > 4));
    System.out.println((3 != 2) || (-1 > 0));
    System.out.println((3 != 2) ^ (-1 > 0));
  }
}
\end{lstlisting}
\section{교재 110p 1번 문제}
\begin{lstlisting}
\lstset{language=Java}
import java.util.Scanner;

public class Prob3 {
  public static void main(String args[]) {
    // Scanner 객체 생성 및 초기화
    Scanner sc = new Scanner(System.in);
    System.out.print("원화를 입력하세요 (단위 원) >> ");
    int gets = sc.nextInt();
    // 문제에서 1100원을 1달러로 제시함
    System.out.println(gets + "원은 $" + (double)gets/1100 + "입니다.");
  }
}
\end{lstlisting}
\section{제곱미터-평 변환 프로그램 작성}
\begin{lstlisting}
\lstset{language=Java}
import java.util.Scanner;

public class Main {
  public static void main(String args[]) {
    System.out.print("제곱미터를 입력하세요 >> ");
    // Scanner 객체 생성 및 초기화
    Scanner sc = new Scanner(System.in);
    int gets = sc.nextInt();

    // 문제에서 1제곱미터는 0.305785평으로 제시함
    System.out.println(gets + "m^2은 " + gets*0.305785 + "평 입니다.");
  }
}
\end{lstlisting}
\section{교재 110p 2번 문제}
\begin{lstlisting}
\lstset{language=Java}
import java.util.Scanner;

public class Main {
  public static void main(String args[]) {
    // Scanner 객체 초기화
    Scanner sc = new Scanner(System.in);
    System.out.print("2자리수 정수 입력 (10~99) >> ");
    int gets = sc.nextInt();
    // gets/10 = 십의 자리 수, gets%10 = 일의 자리 수
    // 문제에서 제시한 조건인 두자리수 
    if (gets/10 == gets%10)
      System.out.println("Yes! 10의 자리와 1의 자리가 같습니다.");
    else
      System.out.println("No! 10의 자리와 1의 자리가 다릅니다.");
  }
}
\end{lstlisting}
\section{Proxima Centauri}
\begin{lstlisting}
\lstset{language=Java}
import java.util.Scanner;
import java.lang.Math;

public class Main {
  public static void main(String args[]) {
    double distance = 40 * Math.pow(10, 12);
    double speed = 3600 * 24 * 365;
    // speed값 선언 시 단위 변환 처리도 함께할 경우
    // double speed = 3600d * 24 * 365 * 300000
    // 와 같이 값 하나를 double형으로 지정하여(3600d)
    // 변수 할당 직전 int형 계산으로 오버플로우가 발생하지 않도록 해야함
    // 그렇지 않으면 정수형 계산에서 발생한 오버플로우의 영향으로
    // double형 변수에 의도한 값이 할당되지 않음
    System.out.println("걸리는 시간은 약 " + distance/(300000*speed) + " 광년 입니다.");
  }
}
\end{lstlisting}
\section{보이저 1호}
\begin{lstlisting}
\lstset{language=Java}
import java.util.Scanner;

public class Main {
  public static void main(String args[]) {
    // 6번 문제 주석에서와 같이 정수형 계산의 오버플로우를 방지하는 차원에서
    // 값을 분할함
    double distance = 40 * Math.pow(10, 12);
    double speed = 24 * 365;
    System.out.println("걸리는 시간은 약 " + distance/(600000*speed) + " 광년 입니다.");
  }
}
\end{lstlisting}
\section{2차 방정식}
\begin{lstlisting}
\lstset{language=Java}
import java.util.Scanner;
import java.lang.Math;

public class Main {
  public static void main(String args[]) {
    // Scanner 객체 초기화
    Scanner sc = new Scanner(System.in);
    int a = sc.nextInt(), b = sc.nextInt(), c = sc.nextInt();
    // 2차방정식의 근의 공식: x = b +_(b^2-4ac)/2a
    System.out.println("x1 = " + (-b+Math.sqrt(Math.pow(b, 2)-4*a*c))/2*a);
    System.out.println("x2 = " + (-b-Math.sqrt(Math.pow(b, 2)-4*a*c))/2*a);
  }
}
\end{lstlisting}
\section{교재 111p 6번 문제}
\begin{lstlisting}
\lstset{language=Java}
import java.util.Scanner;

public class Main {
  public static void main(String args[]) {
    // Scanner 객체 초기화
    Scanner sc = new Scanner(System.in);
    System.out.print("1~99 사이의 정수를 입력하시오 >> ");
    int gets = sc.nextInt();
    int cnt = 0;
    // 일의자리 숫자 처리
    // 일의자리 숫자가 0이 아니고 3으로 나눈 나머지가 0이라면 3/6/9가 포함되었다 볼 수 있음
    if (gets != 0 && (gets % 10) % 3 == 0) cnt += 1;
    System.out.println(gets);
    // 십의자리 숫자 처리
    // 십의자리 숫자가 0이 아니고 3으로 나눈 나머지가 0이라면 3/6/9가 포함되었다 볼 수 있음
    if ((gets / 10) != 0 && (gets / 10) % 3 == 0) cnt += 1;
    // String의 repeat 메서드는 인자값만큼 String 객체를 반복함
    System.out.println("박수" + "짝".repeat(cnt));
  }
}
\end{lstlisting}
\section{교재 111p 9번 문제}
\begin{lstlisting}
\lstset{language=Java}
import java.util.Scanner;
import java.lang.Math;

public class Main {
  public static void main(String args[]) {
    // Scanner 객체 초기화
    Scanner sc = new Scanner(System.in);
    System.out.print("원의 중심과 반지름 입력 >> ");
    double cx = sc.nextDouble();
    double cy = sc.nextDouble();
    double radius = sc.nextDouble();
    System.out.print("점 입력 >> ");
    double x = sc.nextDouble();
    double y = sc.nextDouble();
    // 원의 정의: 한 점으로부터 일정한 거리만큼 떨어져있는 점들의 자취
    // 두 점 사이의 거리 공식 사용
    double dis = Math.sqrt(Math.pow(cx-x, 2) + Math.pow(cy-y, 2));
    // 원의 중심과 원이 반지름보다 멀리 떨어져있는지 체크
    if (dis <= radius)
      System.out.println(String.format("점 (%f, %f)는 원 안에 있다.", x, y));
    else
      System.out.println(String.format("점 (%f, %f)는 원 밖에 있다.", x, y));
  }
}
\end{lstlisting}
\section{교재 112p 11번 문제}
\subsection{if-else문 사용}
\begin{lstlisting}
\lstset{language=Java}
import java.util.Scanner;

public class Main {
  public static void main(String args[]) {
    // Scanner 객체 생성
    Scanner sc = new Scanner(System.in);
    System.out.print("달을 입력하세요(1~12)>>");
    int month = sc.nextInt();
    // if문으로 순차적으로 계절 검출
    // else if 문을 사용하면 조건 표현식을 하나만 작성하여 계절을 확인할 수 있음
    if (month <= 2) System.out.println("겨울");
    else if (month <= 5) System.out.println("봄");
    else if (month <= 8) System.out.println("여름");
    else if (month <= 11) System.out.println("여름");
    else if (month <= 12) System.out.println("겨울");
    else System.out.println("잘못 입력");
  }
}
\end{lstlisting}
\subsection{switch문 사용}
\begin{lstlisting}
\lstset{language=Java}
import java.util.Scanner;

public class Main {
  public static void main(String args[]) {
    // Scanner 객체 생성 및 초기화
    Scanner sc = new Scanner(System.in);
    System.out.print("달을 입력하세요(1~12)>>");
    int month = sc.nextInt();
    switch (month) {
      case 3:
      case 4:
      case 5:
        // switch-case문에서 case는 break를 만나기 전까지
        // 처리를 계속하므로 3 4 5월을 한번에 처리할 수 있음
        System.out.println("봄");
        break;
      case 6:
      case 7:
      case 8:
        // 같은 방식으로 처리
        System.out.println("여름");
        break;
      case 9:
      case 10:
      case 11:
        System.out.println("가을");
        break;
      case 12:
      case 1:
      case 2:
        System.out.println("겨울");
        break;
      default:
        System.out.println("잘못입력");
        break;
    }
  }
}
\end{lstlisting}
\end{document}
